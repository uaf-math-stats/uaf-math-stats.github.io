\documentclass[12 pt]{article}

\usepackage[margin=0.75in]{geometry}
\usepackage[colorlinks]{hyperref}
\usepackage{color}



\begin{document}
	\begin{center}
\textbf{UAF MATH252X -- CALCULUS II}
	\end{center}

\noindent
{\bf Prerequisites:}

Placement into MATH252X is through credit for  MATH251X with a grade of C- or better, or through AP or CLEP exams.
						
\bigskip




\noindent
{\bf Textbook and Course Content:}

Sections of Calculus II offered by UAF campuses (in-person, hybrid, or online) use the textbook \emph{Calculus: Early Transcendentals} by James Stewart, $8^{th}$ edition, available in Single Variable or Single+Multivariable versions.  

Unmarked topics below are taught in all sections. Sections marked with (o) are optional, and may be covered if time permits. 
		
		
		
		
		\begin{itemize}
			\item[]Chapter 6: Applications of Integration
			
			6.1 Areas Between Curves\\
			6.2 Volumes\\
			6.3 Volumes by Cylindrical Shells\\
			6.4 Work\\
			6.5 Average Value of a Function
			\item[]Chapter 7: Techniques of Integration
			
			 7.1 Integration by Parts\\
			 7.2 Trigonometric Integrals\\
			 7.3 Trigonometric Substitution\\
			 7.4 Integration of Rational Functions by Partial Fractions\\
			 7.5 Strategy for Integration\\
			 7.6 Integration using Tables and Computer Algebra Systems (o) \\
			 7.7 Approximate Integration\\
			 7.8 Improper Integrals
			\item[]Chapter 8: Further Applications of Integration
			
			8.1 Arc Length\\
			8.2 Area of a Surface of Revolution\\
			8.3 Applications to Physics and Engineering\\
		        8.4 Applications to Economics and Biology (o)\\
		        8.5 Probability (o)
		         \emph{ \item[]Chapter 9: Differential Equations --- material in Math 302 Differential Equations}
			
			\item[]Chapter 10: Parametric Equations and Polar Coordinates
			 
			 10.1 Curves Defined by Parametric Equations\\
			 10.2 Calculus with Parametric Curves\\
			 10.3 Polar Coordinates\\
			 10.4 Areas and Lengths in Polar Coordinates\\
			 \emph{10.5 Conic Sections --- not covered}\\
			 \emph{10.6 Conic Sections in Polar Coordinates --- not covered}
		         \item[]Chapter 11: Infinite Sequences and Series
		         
		          11.1 Sequences\\
		          11.2 Series \\
		          11.3 The Integral Test and Estimates of Sums\\
		          11.4 The Comparison Tests\\
		          11.5 Alternating Series\\
		          11.6 Absolute Convergence and the Ratio and Root Tests\\
		          11.7 Strategy for Testing Series\\
		          11.8 Power Series\\
		          11.9 Representations of Functions as Power Series\\
		          11.10 Taylor and Maclaurin Series\\
		          11.11 Applications of Taylor Polynomials\\

		\end{itemize}


\bigskip


\noindent
{\bf Assessments and Grading}

Individual instructors may choose to structure their courses within the following guidelines:

	\begin{itemize}
		\item[] Exams: At least two proctored Midterm Exams will be given during the semester, in addition to a cumulative Final Exam. These will be closed book/classnotes. Calculators are not allowed on exams, but questions will not emphasize aspects of problems for which they would be useful. The majority of questions require free responses, and will be graded by course instructors.  Exams will not be reused from previous semesters,  so that they may be provided to students as study materials.
		 
		 
\item[] Other Assessed Work: While on-line homework systems are generally used, courses will include regular assessment through written work graded by an instructor or teaching assistant. Possible forms might include quizzes, written homework problems, worksheets, etc. Instructors may choose to assess oral work. Feedback to students will be given on a roughly weekly basis.



	\item[] The final grade in this course will be determined by weighting assessments within the following ranges:
	
	\begin{center}\begin{tabular}{l|l}
			
			Written Assessed Work & At least 15\% and at most 30\%\\

			\hline
			Online Assessed Work & At most 15\%\\
			\hline
			Midterm Exams & At least 40\% \\
			\hline
			Comprehensive Final Exam & At least 20\%\\
			
			
		\end{tabular}\end{center}
		Instructors may use +/- grading for the course at their discretion.
	\end{itemize}
	
	




\end{document}
