\documentclass[12 pt]{article}

\usepackage[margin=0.75in]{geometry}
\usepackage[colorlinks]{hyperref}
\usepackage{color}



\begin{document}
	\begin{center}
\textbf{UAF MATH253X -- CALCULUS III}
	\end{center}

\noindent
{\bf Prerequisites:}

			Placement into MATH253X is through credit for  MATH252X with a grade of C- or better, or by the BC Calculus AP exam.
						
\bigskip


\noindent
{\bf Textbook and Course Content:}

Sections of Calculus III offered by UAF campuses (in-person, hybrid, or online) use the textbook \emph{Calculus} by James Stewart, $8^{th}$ edition, available in Multivariable and  Single+Multivariable versions.  

Unmarked topics below are taught in all sections. Sections marked with (o) are optional, and may be covered if time permits. 
		
		
		
		
		\begin{itemize}
		
		
		\item[]Chapter 12: Vectors and the Geometry of Space 
		
12.1 Three-Dimensional Coordinate Systems\\
12.2 Vectors \\
12.3 The Dot Product\\ 
12.4 The Cross Product \\
12.5 Equations of Lines and Planes \\
12.6 Cylinders and Quadric Surfaces 
\item[]Chapter 13: Vector Functions

13.1 Vector Functions and Space Curves\\ 
13.2 Derivatives and Integrals of Vector Functions\\ 
13.3 Arc Length and Curvature (curvature (o) )\\ 
13.4 Motion in Space: Velocity and Acceleration
\item[]Chapter 14: Partial Derivatives

14.1 Functions of Several Variables\\ 
14.2 Limits and Continuity\\ 
14.3 Partial Derivatives\\ 
14.4 Tangent Planes and Linear Approximation\\ 
14.5 The Chain Rule\\ 
14.6 Directional Derivatives and the Gradient Vector\\ 
14.7 Maximum and Minimum Values\\ 
14.8 Lagrange Multipliers
\item[]Chapter 15: Multiple Integrals

15.1 Double Integrals over Rectangles\\ 
15.2 Double Integrals over General Regions\\ 
15.3 Double Integrals in Polar Coordinates\\ 
15.4 Applications of Double Integrals\\ 
15.5 Surface Area\\ 
15.6 Triple Integrals\\ 
15.7 Triple Integrals in Cylindrical Coordinates\\ 
15.8 Triple Integrals in Spherical Coordinates\\ 
15.9 Change of Variables in Multiple Integrals (o)
\item[]Chapter 16: Vector Calculus

16.1 Vector Fields\\ 
16.2 Line Integrals\\ 
16.3 The Fundamental Theorem for Line Integrals\\ 
16.4 Green's Theorem\\ 
16.5 Curl and Divergence\\ 
16.6 Parametric Surfaces and Their Areas \\  
16.7 Surface Integrals\\
16.8 Stokes' Theorem\\
16.9 The Divergence Theorem 
\emph{\item[]Chapter 17. Second-order Differential Equations --- material in Math 302 Differential Equations }
		
		
\end{itemize}


\bigskip




\noindent
{\bf Assessments and Grading}

Individual instructors may choose to structure their courses within the following guidelines:

	\begin{itemize}
		\item[] Exams: At least two proctored Midterm Exams will be given during the semester, in addition to a cumulative Final Exam. These will be closed book/classnotes. Calculators are not allowed on exams, but questions will not emphasize aspects of problems for which they would be useful. The majority of questions require free responses, and will be graded by course instructors.  Exams will not be reused from previous semesters,  so that they may be provided to students as study materials.
		 
		 
\item[] Other Assessed Work: While on-line homework systems are generally used, courses will include regular assessment through written work graded by an instructor or teaching assistant. Possible forms might include quizzes, written homework problems, worksheets, etc. Instructors may choose to assess oral work. Feedback to students will be given on a roughly weekly basis.



	\item[] The final grade in this course will be determined by weighting assessments within the following ranges:
	
	\begin{center}\begin{tabular}{l|l}
			
			Written Assessed Work & At least 15\% and at most 30\%\\

			\hline
			Online Assessed Work & At most 15\%\\
			\hline
			Midterm Exams & At least 40\% \\
			\hline
			Comprehensive Final Exam & At least 20\%\\
			
			
		\end{tabular}\end{center}
		Instructors may use +/- grading for the course at their discretion.
	\end{itemize}
	
	




\end{document}
